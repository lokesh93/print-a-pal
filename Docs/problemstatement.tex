\documentclass[12pt]{article}
\usepackage[margin=1in]{geometry}
\usepackage{booktabs}
\usepackage[english]{babel}
\usepackage[utf8x]{inputenc}
\usepackage[T1]{fontenc}


\begin{document}

% Title Page -----------------------------------------------------------------
\title{
\LARGE Print-A-Pal:
\\\vspace{10mm}
\large \textbf{Problem Statement}
\vspace{40mm}
}
\author{
MD Rashad
\\Lokesh Podipireddy
\vspace{10mm}
}
\date{\vfill 2017-04-07}

\pagenumbering{gobble}
\maketitle
\newpage

\begin{table}[h]
\centering
\caption{Revision History}
\begin{tabular}{p{9cm}ll}
\toprule
\textbf{Changes} & \textbf{Author(s)} & \textbf{Date}\\\midrule
Changes made to 2D shape limitations described in first draft due to new 3D rendering method. & MD Rashad & 2017-04-04\\\bottomrule
\end{tabular}
\end{table}

\subsection*{Problem}
There are numerous 3D modeling programs allowing users to create 3D designs in advanced modeling canvases.  To properly print an object in 3D, different printers require different formats and mesh information. The problem with 3D creation and printing currently is that it requires extensive knowledge of the software and 3D modeling techniques that are difficult to learn and combine together to create 3D objects.\\

\noindent Print-A-Pal will allow users to create 3D objects and printable files while allowing them the full freedom and flexibility to create in the 2D canvas as they please.  The simple 2D to 3D rendering method will give users without expertise in 3D design the ability to create their very own real life 3D objects according to their imagination.

\subsection*{Motivation}
The motivation for this project stemmed from curiosity for 3D image rendering and learning about 3D image processing. This project took inspiration from the trending use of 3D printing, and understanding the power and limitations of this field of technology.  The 3D design process continues to become more advanced and intuitive, and thus gives us the capability to do more with the various different hardware or software.\\

\noindent We were interested in working towards bridging the gap between people unfamiliar with the 3D printing process and the formal training or assets needed to accurately produce what was intended to be made.  Creating a software product that provides users with tools to explore and construct 3D designs while allowing them to create in a 2D environment will be a very unique and convenient way for users to get into 3D printing.


\subsection*{Context \& Environment}
The project is designed to be used by primarily by children.  If the product is really successful it may be implemented for use in educational institutions.  The software will be hosted in a creative setting on a website to be used by the compatible web browsers.  Thus, one would need a computer and a secure internet connection to use this software.


 
\end{document}